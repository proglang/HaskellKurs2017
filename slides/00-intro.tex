%% -*- coding: utf-8 -*-
\documentclass{beamer}

\input{common}
%%% frontmatter
%% -*- coding: utf-8 -*-

\title{Programming in Haskell}
\subtitle{Introduction}

\author[Peter Thiemann]{Prof. Dr. Peter Thiemann}
\institute[Univ. Freiburg]{Albert-Ludwigs-Universität Freiburg, Germany}
\date[HaskellKurs]{Sommercampus 2017}


\usepackage{tikz}


\begin{document}

\begin{frame}
  \titlepage
\end{frame}

\begin{frame}[fragile]
  \frametitle{Coordinates}
  \begin{itemize}
  \item \textbf{Course hours:}  Th, 9-18 Uhr, HS 101-00-026
  \item \textbf{Staff:}  Alexander Thiemann, Prof.\ Dr.\ Peter Thiemann,
    Dr.\ Stefan Wehr\\
\begin{alltt}
Gebäude 079, Raum 00-015 
Telefon: 0761 203 -8051/-8247
E-mail: thiemann@cs.uni-freiburg.de
Web: \href{http://www.informatik.uni-freiburg.de/~thiemann}{http://www.informatik.uni-freiburg.de/~thiemann}
\end{alltt}
  \item\textbf{Homepage:}\\ \footnotesize
    \href{https://github.com/proglang/HaskellKurs2017}{
      https://github.com/proglang/HaskellKurs2017}
  \end{itemize}
\end{frame}

%----------------------------------------------------------------------

\begin{frame}
  \frametitle{Contents}
  \begin{itemize}
  \item Basics of functional programming using Haskell
  \item Haskell development tools
  \item Writing Haskell programs
  \item Using Haskell libraries
  \item Your first Haskell project
  \end{itemize}
\end{frame}


%----------------------------------------------------------------------
\begin{frame}
  \frametitle{What is Haskell?}
  \begin{quotation}
    In September of 1987 a meeting was held at the conference on
    Functional Programming Languages and Computer Architecture in
    Portland, Oregon, to discuss an unfortunate situation in the
    functional programming community: there had come into being more
    than a dozen non-strict, purely functional programming languages,
    all similar in expressive power and semantic underpinnings. There
    was a strong consensus at this meeting that more widespread use
    of this class of functional languages was being hampered by the
    lack of a common language. It was decided that a committee should
    be formed to design such a language, providing faster
    communication of new ideas, a stable foundation for real
    applications development, and a vehicle through which others would 
    be encouraged to use functional languages. 
  \end{quotation}
  {\tiny From ``History of Haskell''}
\end{frame}

%----------------------------------------------------------------------
\begin{frame}
  \frametitle{What is Functional Programming?}
  \begin{block}<+->{A different approach to programming}
    \begin{LARGE}
      \begin{center}
        Functions and values
        \\[3ex]
        rather than
        \\[3ex]
        Assignments and pointers
      \end{center}
    \end{LARGE}
  \end{block}
  \begin{alertblock}<+->{It will make you a better programmer}
    
  \end{alertblock}
\end{frame}
%----------------------------------------------------------------------
\begin{frame}
  \frametitle{Why Haskell?}
  \begin{itemize}
  \item Haskell is a very high-level language
    \\\hfill\textcolor{gray}{(many details taken care
    of automatically)}. 
  \item Haskell is expressive and concise
    \\\hfill\textcolor{gray}{(can achieve a lot with a
    little effort)}. 
  \item Haskell is good at handling complex data and combining
    components. 
  \item Haskell is a high-productivity language \\\hfill\textcolor{gray}{ (prioritize programmer-time over computer-time)}
  \end{itemize}
\end{frame}
%----------------------------------------------------------------------

\begin{frame}[fragile]
  \frametitle{Functional Programming: Variables}
  \begin{block}<+->{Functional (Haskell)}
\begin{verbatim}
x :: Int
x = 5
\end{verbatim}
    Variable \texttt{x} has value \texttt{5} forever
  \end{block}
  \begin{block}<+->{Imperative (Java)}
\begin{verbatim}
int x = 5;
...
x = x+1;
\end{verbatim}
    Variable \texttt{x} can change its content over time
  \end{block}
\end{frame}

\begin{frame}[fragile]
  \frametitle{Functional Programming: Functions}
  \begin{block}<+->{Functional (Haskell)}
\begin{verbatim}
f :: Int -> Int -> Int
f x y = 2*x + y

f 42 16 // always 100
\end{verbatim}
    Value of a function \textbf{only} depends on its inputs
  \end{block}
  \begin{block}<+->{Imperative (Java)}
\begin{verbatim}
boolean flag;
static int f (int x, int y) {
  return flag ? 2*x + y , 2*x - y;
}

f (42, 16); // who knows?
\end{verbatim}
    Return value  depends on non-local variable \texttt{flag}
  \end{block}
\end{frame}

\begin{frame}[fragile]
  \frametitle{Functional Programming: Laziness}
  \begin{block}<+->{Haskell}
\begin{verbatim}
x = expensiveComputation
g anotherExpensiveComputation
\end{verbatim}
    \begin{itemize}
    \item The expensive computation will only happen if \texttt{x} is
      ever used.
    \item Another expensive computation will only happen if \texttt{g}
      uses its argument.
    \end{itemize}
  \end{block}
  \begin{block}<+->{Java}
\begin{verbatim}
int x = expensiveComputation;
g (anotherExpensiveComputation)
\end{verbatim}
    Both expensive computations will happen anyway.
  \end{block}
\end{frame}

%----------------------------------------------------------------------
\begin{frame}
  \frametitle{Many more features that make programs more concise}
  \begin{itemize}
  \item Algebraic datatypes
  \item Polymorphic types
  \item Parametric overloading
  \item Type inference
  \item Monads \& friends (for IO, concurrency, \dots)
  \item Comprehensions
  \item Metaprogramming
  \item Domain specific languages
  \item \dots
  \end{itemize}
\end{frame}
%----------------------------------------------------------------------
\begin{frame}
  \frametitle{References}
  \begin{itemize}
  \item Paper by the original developers of Haskell in the conference on History of
    Programming Languages (HOPL III): \href{A History
      of Haskell: Being Lazy with
      Class}{http://dl.acm.org/citation.cfm?id=1238856}
  \item The Haskell home page: \url{http://www.haskell.org}
  \item Haskell libraries repository:
    \url{https://hackage.haskell.org/}
  \item Haskell Tool Stack: \url{https://docs.haskellstack.org/en/stable/README/}
  \end{itemize}
\end{frame}

%----------------------------------------------------------------------

\begin{frame}
  % \frametitle{Questions?}
  \begin{center}
    \tikz{\node[scale=15] at (0,0){?};}
  \end{center}
\end{frame}

%----------------------------------------------------------------------

\begin{frame}
  \begin{Huge}
    \begin{center}
      Let's get started!
    \end{center}
  \end{Huge}
\end{frame}

%----------------------------------------------------------------------

\begin{frame}[fragile]
  \frametitle{Haskell Demo}
  \begin{itemize}
  \item  Let's say we want to buy a game in the USA and we have to
    convert its price from USD to EUR
  \item  A \textbf{definition} gives a name to a value
  \item Names are  case-sensitive, must start with lowercase letter
  \item Definitions are  put in a text file ending in \texttt{.hs}
  \end{itemize}
  \begin{block}{Examples.hs}
\begin{verbatim}
dollarRate = 1.3671
\end{verbatim}
  \end{block}
\end{frame}
\begin{frame}[fragile]
  \frametitle{Using the definition}
  \begin{itemize}
  \item Start the Haskell interpreter  GHCi\\
    \texttt{> stack ghci}
\begin{verbatim}
Configuring GHCi with the following packages: 
GHCi, version 8.0.1: http://www.haskell.org/ghc/  :? for help
Loaded GHCi configuration from /private/var/folders/f1/7mmh0sps717gj4hgppqf20zc0000gp/T/ghci58527/ghci-script
Prelude> 
\end{verbatim}
  \item Load the file\\
    \texttt{Prelude> :l Examples.hs}
\begin{verbatim}
[1 of 1] Compiling Main             ( Examples.hs, interpreted )
Ok, modules loaded: Main.
*Main> 
\end{verbatim}
  \item Use the definition
\begin{verbatim}
*Main> dollarRate
1.3671
*Main> 53 * dollarRate
72.4563
\end{verbatim}
  \end{itemize}
\end{frame}
\begin{frame}[fragile]
  \frametitle{A function to convert EUR to USD}
\begin{block}{Examples.hs}
\begin{verbatim}
dollarRate = 1.3671

-- |convert EUR to USD
usd euros = euros * dollarRate
\end{verbatim}
  \end{block}
  \begin{itemize}
  \item line starting with \texttt{--}: comment
  \item \texttt{usd}: function name (defined)
  \item \texttt{euros}:  argument name (defined)
  \item \texttt{euros * dollarRate}: expression to compute the result
  \end{itemize}
\end{frame}
\begin{frame}[fragile]
  \frametitle{Using the function}
  \begin{itemize}
  \item load into GHCi
    \begin{itemize}
    \item as before or
    \item use \texttt{:r} to reload
    \end{itemize}
  \end{itemize}
\begin{verbatim}
*Main> usd 1
1.3671
*Main> usd 73
99.7983
\end{verbatim}
\end{frame}
\begin{frame}[fragile]
  \frametitle{Converting back}
Write a function \texttt{euro} that converts back from USD to EUR!
\begin{verbatim}
*Main> euro (usd 73)
73.0
*Main> euro (usd 1)
1.0
*Main> usd (euro 100)
100.0
\end{verbatim}
\begin{alertblock}<2->{Your turn}

\end{alertblock}
\end{frame}

\begin{frame}[fragile]
  \frametitle{Testing properties}
  \framesubtitle{Is this function correct?}
  \begin{block}<+->{A reasonable property of \texttt{euro} and \texttt{usd}}
\begin{verbatim}
prop_EuroUSD x = euro (usd x) == x
\end{verbatim}
    \texttt{==} is the equality operator
\begin{verbatim}
*Main> prop_EuroUSD 79
True
*Main> prop_EuroUSD 1
True
\end{verbatim}
  \end{block}
  \begin{alertblock}<+->{Does it hold?}
    
  \end{alertblock}
\end{frame}
\begin{frame}[fragile]
  \frametitle{Aside: Writing Properties}
  \begin{block}{Convention}
    Function names beginning with
    \verb|prop_| are properties we expect to be True 
  \end{block}
  \begin{block}{Writing properties in a file}
    \begin{itemize}
    \item Tells us how functions should behave 
    \item Tells us what has been tested 
    \item Lets us repeat tests after changing a definition 
    \end{itemize}
  \end{block}
\end{frame}

\begin{frame}[fragile]
  \frametitle{Testing}
  \begin{block}{At the beginning of Examples.hs}
\begin{verbatim}
import Test.QuickCheck
\end{verbatim}
    A widely used Haskell library for automatic random testing
  \end{block}
\end{frame}
\begin{frame}[fragile]
  \frametitle{Running tests}
\begin{verbatim}
*Main> quickCheck prop_EuroUSD
*** Failed! Falsifiable (after 10 tests and 1 shrink): 
7.0
\end{verbatim}
  \begin{itemize}
  \item Runs 100 randomly chosen tests
  \item Result: The property is wrong!
  \item It fails for input 7.0
  \end{itemize}
  \begin{alertblock}<2->{Check what happens for 7.0!}
    
  \end{alertblock}
\end{frame}
\begin{frame}[fragile]
  \frametitle{What happens for 7.0}
\begin{verbatim}
*Main> usd 7
9.5697
*Main> euro 9.5697
6.999999999999999
\end{verbatim}
\end{frame}
\begin{frame}[fragile]
  \frametitle{The Problem: Floating Point Arithmetic}
  \begin{itemize}
  \item  There is a very tiny difference between the initial and final values 
\begin{verbatim}
*Main> euro (usd 7) - 7
-8.881784197001252e-16
\end{verbatim}
  \item Calculations are only performed to about 15 significant
    figures 
  \item  The property is wrong! 
  \end{itemize}
\end{frame}
\begin{frame}
  \frametitle{Fixing the problem}
  \begin{itemize}
  \item NEVER use equality with floating point numbers!
  \item  The result should be \emph{nearly} the same 
  \item  The difference should be small – smaller than 10E-15
  \end{itemize}
\end{frame}
\begin{frame}[fragile]
  \frametitle{Comparing Values}
\begin{verbatim}
*Main> 2<3
True
*Main> 3<2
False
\end{verbatim}
\end{frame}
\begin{frame}[fragile,fragile]
  \frametitle{Defining ``Nearly Equal''}
  \begin{itemize}
  \item Can define new operators with names made up of symbols
  \end{itemize}
\begin{block}{In Examples.hs}
\begin{verbatim}
x ~== y = abs(x - y) < 10e-15 * abs x
\end{verbatim}
  \end{block}
\begin{verbatim}
*Main> 3 ~== 3.0000001
True
*Main> 3 ~== 4
True
\end{verbatim}
\end{frame}
\begin{frame}[fragile]
  \frametitle{Fixing the property}
\begin{block}{In Examples.hs}
\begin{verbatim}
prop_EuroUSD' x = euro (usd x) ~== x
\end{verbatim}
  \end{block}
\begin{verbatim}
*Main> prop_EuroUSD' 3
True
*Main> prop_EuroUSD' 56
True
*Main> prop_EuroUSD' 7
True
\end{verbatim}
\end{frame}

\begin{frame}[fragile]
  \frametitle{Name the price}
  \begin{block}<+->{Let's define a name for the price of the game we want
      in Examples.hs}
\begin{verbatim}
price = 79
\end{verbatim}
  \end{block}
  \begin{alertblock}<+->{After reload: Ouch!}
\begin{verbatim}
*Main> euro price

<interactive>:57:6:
    Couldn't match expected type `Double' with actual type `Integer'
    In the first argument of `euro', namely `price'
    In the expression: euro price
    In an equation for `it': it = euro price
\end{verbatim}
  \end{alertblock}
\end{frame}
\begin{frame}[fragile]
  \frametitle{Every Value has a type}
  The \texttt{:i} command prints information about a nae
\begin{verbatim}
*Main> :i price
price :: Integer
  	-- Defined at ...
*Main> :i dollarRate
dollarRate :: Double
  	-- Defined at ...
\end{verbatim}
\end{frame}
\begin{frame}[fragile]
  \frametitle{More types}
\begin{verbatim}
*Main> :i True
data Bool = ... | True 	-- Defined in `GHC.Types'
*Main> :i False
data Bool = False | ... 	-- Defined in `GHC.Types'
*Main> :i euro
euro :: Double -> Double
  	-- Defined at...
*Main> :i prop_EuroUSD'
prop_EuroUSD' :: Double -> Bool
  	-- Defined at...
\end{verbatim}
  \begin{itemize}
  \item \texttt{True} and \texttt{False} are \textbf{data constructors}
  \end{itemize}
\end{frame}
\begin{frame}[fragile]
  \frametitle{Types matter}
  \begin{itemize}
  \item Types determine how computations are performed
  \item A type annotation specifies which type to use
  \end{itemize}
\begin{verbatim}
*Main> 123456789*123456789 :: Double
1.524157875019052e16
*Main> 123456789*123456789 :: Integer
15241578750190521
\end{verbatim}
  \begin{itemize}
  \item \texttt{Double}: double precision floating point 
  \item  \texttt{Integer}: exact computation
  \item  GHCi must know the type of each expression before computing it.
  \end{itemize}
\end{frame}
\begin{frame}
  \frametitle{Type checking}
  \begin{itemize}
  \item  Infers (works out) the type of every expression 
  \item  Checks that all types match --- before running the program
  \end{itemize}
\end{frame}
\begin{frame}[fragile]
  \frametitle{Our example}
\begin{verbatim}
*Main> :i price
price :: Integer
  	-- Defined at...
*Main> :i euro
euro :: Double -> Double
  	-- Defined at...
*Main> euro price

<interactive>:70:6:
    Couldn't match expected type `Double' with actual type `Integer'
    In the first argument of `euro', namely `price'
    In the expression: euro price
    In an equation for `it': it = euro price
\end{verbatim}
\end{frame}
\begin{frame}[fragile]
  \frametitle{Why did it work before?}
  \begin{itemize}
  \item Numeric literals are \textbf{overloaded}
  \item  Giving the number a name fixes its type
  \end{itemize}
\begin{verbatim}
*Main> euro 79
57.78655548240802
*Main> 79 :: Integer
79
*Main> 79 :: Double
79.0
*Main> price :: Integer
79
*Main> price :: Double

<interactive>:76:1:
    Couldn't match expected type `Double' with actual type `Integer'
    In the expression: price :: Double
    In an equation for `it': it = price :: Double
\end{verbatim}
\end{frame}
\begin{frame}[fragile]
  \frametitle{Fixing the problem}
  A definition can be given a \textbf{type signature} which specifies
  its type
\begin{block}{In Examples.hs}
\begin{verbatim}
-- |price of the game in USD
price' :: Double
price' = 79
\end{verbatim}
  \end{block}
\begin{verbatim}
*Main> :i price'
price' :: Double
  	-- Defined at...
*Main> euro price'
57.78655548240802
\end{verbatim}
\end{frame}

%----------------------------------------------------------------------

\begin{frame}
  \frametitle{Break Time --- Questions?}
  \begin{center}
    \tikz{\node[scale=15] at (0,0){?};}
  \end{center}
\end{frame}


\end{document}

%%% Local Variables: 
%%% mode: latex
%%% TeX-master: t
%%% End: 
