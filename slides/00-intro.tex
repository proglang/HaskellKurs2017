%% -*- coding: utf-8 -*-
\documentclass{beamer}

%% -*- coding: utf-8 -*-
\usetheme{Boadilla} % default
\useoutertheme{infolines}
\setbeamertemplate{navigation symbols}{} 

\usepackage{etex}
\usepackage{alltt}
\usepackage{pifont}
\usepackage{color}
\usepackage[utf8]{inputenc}
\usepackage{german}
\usepackage{listings}
\usepackage{hyperref}
\usepackage[final]{pdfpages}
\usepackage{url}
\usepackage{arydshln} % dashed lines

\newcommand\cmark{\ding{51}}
\newcommand\xmark{\ding{55}}

\newcommand{\nat}{\mathbf{N}}

\usepackage[all]{xy}

%% new arrow tip for xy
\newdir{|>}{!/4.5pt/@{|}*:(1,-.2)@^{>}*:(1,+.2)@_{>}}

\newcommand\cid[1]{\textup{\textbf{#1}}} % class names
\newcommand\kw[1]{\textup{\textbf{#1}}}  % key words
\newcommand\tid[1]{\textup{\textsf{#1}}} % type names
\newcommand\vid[1]{\textup{\texttt{#1}}} % value names
\newcommand\Mid[1]{\textup{\texttt{#1}}} % method names

\newcommand\TODO[1][]{{\color{red}{\textbf{TODO: #1}}}}

\newcommand\String[1]{\texttt{\dq{}#1\dq{}}}

\newcommand\ClassHead[1]{%
  \ensuremath{\begin{array}{|l|}
      \hline
      \cid{#1}
      \\\hline
    \end{array}}}
\newcommand\AbstractClass[2]{%
  \ensuremath{\begin{array}{|l|}
      \hline
      \cid{\textit{#1}}
      \\\hline
      #2
      \hline
    \end{array}}}
\newcommand\Class[2]{%
  \ensuremath{\begin{array}{|l|}
      \hline
      \cid{#1}
      \\\hline
      #2
      \hline
    \end{array}}}
\newcommand\Attribute[3][black]{\textcolor{#1}{\Param{#2}{#3}}\\}
\newcommand\Methods{\hline}
\newcommand\MethodSig[3]{\Mid{#2} (#3): \,\tid{#1}\\}
\newcommand\CtorSig[2]{\Mid{#1} (#2)\\}
\newcommand\AbstractMethodSig[3]{\Mid{\textit{#2}} (#3): \,\tid{#1}\\}
\newcommand\Param[2]{\vid{#2}:~\tid{#1}}

\lstset{%
  frame=single,
  xleftmargin=2pt,
  stepnumber=1,
  numbers=left,
  numbersep=5pt,
  numberstyle=\ttfamily\tiny\color[gray]{0.3},
  belowcaptionskip=\bigskipamount,
  captionpos=b,
  escapeinside={*'}{'*},
  language=java,
  tabsize=2,
  emphstyle={\bf},
  commentstyle=\mdseries\it,
  stringstyle=\mdseries\rmfamily,
  showspaces=false,
  showtabs=false,
  keywordstyle=\bfseries,
  columns=fullflexible,
  basicstyle=\footnotesize\CodeFont,
  showstringspaces=false,
  morecomment=[l]\%,
  rangeprefix=////,
  includerangemarker=false,
}

\newcommand\CodeFont{\sffamily}

\definecolor{lightred}{rgb}{0.8,0,0}
\definecolor{darkgreen}{rgb}{0,0.5,0}
\definecolor{darkblue}{rgb}{0,0,0.5}

\newcommand\highlight[1]{\textcolor{blue}{\emph{#1}}}
\newcommand\GenClass[2]{\cid{#1}\texttt{<}\cid{#2}\texttt{>}}

\newcommand\Colored[3]{\alt<#1>{\textcolor{#2}{#3}}{#3}}

\newcommand\nt[1]{\ensuremath{\langle#1\rangle}}

%%% Local Variables: 
%%% mode: latex
%%% TeX-master: nil
%%% End: 

%%% frontmatter
%% -*- coding: utf-8 -*-

\title{Programming in Haskell}
\subtitle{Introduction}

\author[Peter Thiemann]{Prof. Dr. Peter Thiemann}
\institute[Univ. Freiburg]{Albert-Ludwigs-Universität Freiburg, Germany}
\date[HaskellKurs]{Sommercampus 2017}


\usepackage{tikz}


\begin{document}

\begin{frame}
  \titlepage
\end{frame}

\begin{frame}[fragile]
  \frametitle{Coordinates}
  \begin{itemize}
  \item \textbf{Course hours:}  Th, 9-18 Uhr, HS 101-00-026
  \item \textbf{Staff:}  Alexander Thiemann, Prof.\ Dr.\ Peter Thiemann,
    Dr.\ Stefan Wehr\\
\begin{alltt}
Gebäude 079, Raum 00-015 
Telefon: 0761 203 -8051/-8247
E-mail: thiemann@cs.uni-freiburg.de
Web: \href{http://www.informatik.uni-freiburg.de/~thiemann}{http://www.informatik.uni-freiburg.de/~thiemann}
\end{alltt}
  \item\textbf{Homepage:}\\ \footnotesize
    \href{https://github.com/proglang/HaskellKurs2017}{
      https://github.com/proglang/HaskellKurs2017}
  \end{itemize}
\end{frame}

%----------------------------------------------------------------------

\begin{frame}
  \frametitle{Contents}
  \begin{itemize}
  \item Basics of functional programming using Haskell
  \item Haskell development tools
  \item Writing Haskell programs
  \item Using Haskell libraries
  \item Your first Haskell project
  \end{itemize}
\end{frame}


%----------------------------------------------------------------------
\begin{frame}
  \frametitle{What is Haskell?}
  \begin{quotation}
    In September of 1987 a meeting was held at the conference on
    Functional Programming Languages and Computer Architecture in
    Portland, Oregon, to discuss an unfortunate situation in the
    functional programming community: there had come into being more
    than a dozen non-strict, purely functional programming languages,
    all similar in expressive power and semantic underpinnings. There
    was a strong consensus at this meeting that more widespread use
    of this class of functional languages was being hampered by the
    lack of a common language. It was decided that a committee should
    be formed to design such a language, providing faster
    communication of new ideas, a stable foundation for real
    applications development, and a vehicle through which others would 
    be encouraged to use functional languages. 
  \end{quotation}
  {\tiny From ``History of Haskell''}
\end{frame}

%----------------------------------------------------------------------
\begin{frame}
  \frametitle{What is Functional Programming?}
  \begin{LARGE}
    \begin{center}
      Functions and values
      \\[4ex]
      rather than
      \\[4ex]
      Assignments and pointers
    \end{center}
  \end{LARGE}
\end{frame}

\begin{frame}[fragile]
  \frametitle{Functional Programming: Variables}
  \begin{block}<+->{Functional (Haskell)}
\begin{verbatim}
x :: Int
x = 5nn
\end{verbatim}
    Variable \texttt{x} has value \texttt{5} forever
  \end{block}
  \begin{block}<+->{Imperative (Java)}
\begin{verbatim}
int x = 5;
...
x = x+1;
\end{verbatim}
    Variable \texttt{x} can change its content over time
  \end{block}
\end{frame}

\begin{frame}[fragile]
  \frametitle{Functional Programming: Functions}
  \begin{block}<+->{Functional (Haskell)}
\begin{verbatim}
f :: Int -> Int -> Int
f x y = 2*x + y

f 42 16 // always 100
\end{verbatim}
    Value of a function \textbf{only} depends on its inputs
  \end{block}
  \begin{block}<+->{Imperative (Java)}
\begin{verbatim}
boolean flag;
static int f (int x, int y) {
  return flag ? 2*x + y , 2*x - y;
}

f (42, 16); // who knows?
\end{verbatim}
    Return value  depends on non-local variable
  \end{block}
\end{frame}

\begin{frame}[fragile]
  \frametitle{Functional Programming: Laziness}
  \begin{block}<+->{Haskell}
\begin{verbatim}
x = expensiveComputation
g anotherExpensiveComputation
\end{verbatim}
    \begin{itemize}
    \item The expensive computation will only happen if \texttt{x} is
      ever used.
    \item Another expensive computation will only happen if \texttt{g}
      uses its argument.
    \end{itemize}
  \end{block}
  \begin{block}<+->{Java}
\begin{verbatim}
int x = expensiveComputation;
g (anotherExpensiveComputation)
\end{verbatim}
    Both expensive computations will happen anyway.
  \end{block}
\end{frame}

%----------------------------------------------------------------------
\begin{frame}
  \frametitle{Many more features that make programs more concise}
  \begin{itemize}
  \item Algebraic datatypes
  \item Polymorphic types
  \item Parametric overloading
  \item Type inference
  \item Monads \& friends (for IO, concurrency, \dots)
  \item Comprehensions
  \item Metaprogramming
  \item Domain specific languages
  \item \dots
  \end{itemize}
\end{frame}
%----------------------------------------------------------------------
\begin{frame}
  \frametitle{References}
  \begin{itemize}
  \item Paper by the original developers of Haskell in the conference on History of
    Programming Languages (HOPL III): \href{A History
      of Haskell: Being Lazy with
      Class}{http://dl.acm.org/citation.cfm?id=1238856}
  \item The Haskell home page: \url{http://www.haskell.org}
  \item Haskell libraries repository:
    \url{https://hackage.haskell.org/}
  \item Haskell Tool Stack: \url{https://docs.haskellstack.org/en/stable/README/}
  \end{itemize}
\end{frame}

%----------------------------------------------------------------------

\begin{frame}
  % \frametitle{Questions?}
  \begin{center}
    \tikz{\node[scale=15] at (0,0){?};}
  \end{center}
\end{frame}

%----------------------------------------------------------------------

\begin{frame}
  \begin{Huge}
    \begin{center}
      Let's get started!
    \end{center}
  \end{Huge}
\end{frame}
\end{document}
%%% Local Variables: 
%%% mode: latex
%%% TeX-master: t
%%% End: 
