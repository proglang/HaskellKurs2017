%% -*- coding: utf-8 -*-
\documentclass{beamer}

%% -*- coding: utf-8 -*-
\usetheme{Boadilla} % default
\useoutertheme{infolines}
\setbeamertemplate{navigation symbols}{} 

\usepackage{etex}
\usepackage{alltt}
\usepackage{pifont}
\usepackage{color}
\usepackage[utf8]{inputenc}
\usepackage{german}
\usepackage{listings}
\usepackage{hyperref}
\usepackage[final]{pdfpages}
\usepackage{url}
\usepackage{arydshln} % dashed lines

\newcommand\cmark{\ding{51}}
\newcommand\xmark{\ding{55}}

\newcommand{\nat}{\mathbf{N}}

\usepackage[all]{xy}

%% new arrow tip for xy
\newdir{|>}{!/4.5pt/@{|}*:(1,-.2)@^{>}*:(1,+.2)@_{>}}

\newcommand\cid[1]{\textup{\textbf{#1}}} % class names
\newcommand\kw[1]{\textup{\textbf{#1}}}  % key words
\newcommand\tid[1]{\textup{\textsf{#1}}} % type names
\newcommand\vid[1]{\textup{\texttt{#1}}} % value names
\newcommand\Mid[1]{\textup{\texttt{#1}}} % method names

\newcommand\TODO[1][]{{\color{red}{\textbf{TODO: #1}}}}

\newcommand\String[1]{\texttt{\dq{}#1\dq{}}}

\newcommand\ClassHead[1]{%
  \ensuremath{\begin{array}{|l|}
      \hline
      \cid{#1}
      \\\hline
    \end{array}}}
\newcommand\AbstractClass[2]{%
  \ensuremath{\begin{array}{|l|}
      \hline
      \cid{\textit{#1}}
      \\\hline
      #2
      \hline
    \end{array}}}
\newcommand\Class[2]{%
  \ensuremath{\begin{array}{|l|}
      \hline
      \cid{#1}
      \\\hline
      #2
      \hline
    \end{array}}}
\newcommand\Attribute[3][black]{\textcolor{#1}{\Param{#2}{#3}}\\}
\newcommand\Methods{\hline}
\newcommand\MethodSig[3]{\Mid{#2} (#3): \,\tid{#1}\\}
\newcommand\CtorSig[2]{\Mid{#1} (#2)\\}
\newcommand\AbstractMethodSig[3]{\Mid{\textit{#2}} (#3): \,\tid{#1}\\}
\newcommand\Param[2]{\vid{#2}:~\tid{#1}}

\lstset{%
  frame=single,
  xleftmargin=2pt,
  stepnumber=1,
  numbers=left,
  numbersep=5pt,
  numberstyle=\ttfamily\tiny\color[gray]{0.3},
  belowcaptionskip=\bigskipamount,
  captionpos=b,
  escapeinside={*'}{'*},
  language=java,
  tabsize=2,
  emphstyle={\bf},
  commentstyle=\mdseries\it,
  stringstyle=\mdseries\rmfamily,
  showspaces=false,
  showtabs=false,
  keywordstyle=\bfseries,
  columns=fullflexible,
  basicstyle=\footnotesize\CodeFont,
  showstringspaces=false,
  morecomment=[l]\%,
  rangeprefix=////,
  includerangemarker=false,
}

\newcommand\CodeFont{\sffamily}

\definecolor{lightred}{rgb}{0.8,0,0}
\definecolor{darkgreen}{rgb}{0,0.5,0}
\definecolor{darkblue}{rgb}{0,0,0.5}

\newcommand\highlight[1]{\textcolor{blue}{\emph{#1}}}
\newcommand\GenClass[2]{\cid{#1}\texttt{<}\cid{#2}\texttt{>}}

\newcommand\Colored[3]{\alt<#1>{\textcolor{#2}{#3}}{#3}}

\newcommand\nt[1]{\ensuremath{\langle#1\rangle}}

%%% Local Variables: 
%%% mode: latex
%%% TeX-master: nil
%%% End: 

%%% frontmatter
%% -*- coding: utf-8 -*-

\title{Programming in Haskell}
\subtitle{Introduction}

\author[Peter Thiemann]{Prof. Dr. Peter Thiemann}
\institute[Univ. Freiburg]{Albert-Ludwigs-Universität Freiburg, Germany}
\date[HaskellKurs]{Sommercampus 2017}


\subtitle{Types}
\usepackage{tikz}


\begin{document}

\begin{frame}
  \titlepage
\end{frame}
%----------------------------------------------------------------------
\begin{frame}[fragile]
  \frametitle{Predefined Types}
  \begin{itemize}
  \item \texttt{Bool} (True, False)
  \item  \texttt{Char} ('x', '?', \dots)
  \item \texttt{Double}, \texttt{Float} 
  \item   \texttt{Integer}
  \item  \texttt{Int} --- machine integers ($\ge$ 30 bits signed
    integer)
  \item  ()
    --- the unit type, single value () 
  \item  function types
  \item tuples and lists
  \item \texttt{String} (\verb|"xyz"|, \dots)  
  \item \dots
  \end{itemize}
\end{frame}
%----------------------------------------------------------------------
\begin{frame}[fragile]
  \frametitle{Tuples}
  \begin{block}{}
\begin{verbatim}
-- example tuples
examplePair :: (Double, Bool)  -- Double x Bool
examplePair = (3.14, False)

exampleTriple :: (Bool, Int, String) -- Bool x Int x String
exampleTriple = (False, 42, "Answer")

exampleFunction :: (Bool, Int, String) -> Bool
exampleFunction (b, i, s) = not b && length s < i
\end{verbatim}
  \end{block}
  \begin{alertblock}{Summary}
    \begin{itemize}
    \item syntax for tuple type like syntax for tuple values
    \item tuples are \textbf{immutable} : once a tuple value is defined it cannot change!
    \end{itemize}
  \end{alertblock}
\end{frame}
%----------------------------------------------------------------------
\begin{frame}
  \frametitle{Lists}
  \begin{itemize}
  \item The “duct tape” of functional programming
  \item Collections of things of the same type 
  \item 
    For any type \texttt{x}, \texttt{[x]} is the type of lists of \texttt{x}s
    \\ e.g. \texttt{[Bool]} is the type of lists of \texttt{Bool}
  \item  syntax for list type like syntax for list values
  \item lists are \textbf{immutable} : once a list value is defined it cannot change!
  \end{itemize}
\end{frame}
%----------------------------------------------------------------------
\begin{frame}[fragile]
  \frametitle{Constructing lists}
  \begin{block}<+->{The values of [a] are \dots}
    \begin{itemize}
    \item either \texttt{[]}, the empty list
    \item or \texttt{x:xs} where \texttt{x} has type \texttt{a} and
      \texttt{xs} has type \texttt{[a]} \\
      ``\texttt{:}'' is pronounced ``cons''
    \end{itemize}
  \end{block}
  \begin{block}<+->{Quiz}
    Which of the following expressions have type \texttt{[Bool]}?
\begin{verbatim}
  []
  True : [ ]
  True:False
  False:(False:[ ])
  (False:False):[ ]
  (False:[]):[ ]
  (True : (False : (True : []))) : (False:[]):[ ]
\end{verbatim}
  \end{block}
\end{frame}
%----------------------------------------------------------------------
\begin{frame}
  \frametitle{List shorthands}
  \begin{block}{Equivalent ways of writing a list}
    \begin{itemize}
    \item \texttt{1:(2:(3:[ ]))}  --- standard, fully parenthesized
    \item \texttt{1:2:3:[ ]} --- \texttt{(:)} associates to the right
    \item \texttt{[1,2,3]}
    \end{itemize}
  \end{block}
\end{frame}
%----------------------------------------------------------------------
\begin{frame}[fragile]
  \frametitle{Functions on lists}
  \begin{block}<+->{Definition by \textbf{pattern matching}}
\begin{verbatim}
-- function over lists - examples
summerize :: [String] -> String
summerize []  = "None"
summerize [x] = "Only " ++ x
summerize [x,y] = "Two things: " ++ x ++ " and " ++ y
summerize [_,_,_] = "Three things: ???"
summerize _   = "Several things."   -- wild card pattern
\end{verbatim}
  \end{block}
\end{frame}
\begin{frame}
  \frametitle{List function with pattern matching}
\begin{alertblock}<+->{Explanations}
    \small
    \begin{itemize}
    \item \texttt{(++)} \textbf{list concatenation}, associates to right because it's more efficient
      \begin{itemize}
      \item \texttt{ [1,2,3,4,5] ++ ([6,7,8,9] ++ [])} --- 10 copy
        operations
      \item 
        \texttt{([1,2,3,4,5] ++ [6,7,8,9]) ++ []} --- 14 copy operations,
        because \texttt{[1,2,3,4,5]} is copied twice
      \end{itemize}
    \item  patterns are checked in sequence
    \item  variables in patterns are bound to the values in
      corresponding position in the argument
    \item  each variable may occur at most once in a pattern
    \item  wild card pattern \verb|_| matches everything, no binding, may occur multiple times
    \end{itemize}
  \end{alertblock}
\end{frame}
%----------------------------------------------------------------------
\begin{frame}[fragile]
  \frametitle{Primitive recursion on lists}
  \begin{block}{Common example}
\begin{verbatim}
-- doubles [3,6,12] = [6,12,24]
doubles :: [Integer] -> [Integer]
doubles []     = undefined
doubles (x:xs) = undefined
\end{verbatim}
    BUT: Would not write it in this way --- it's a common pattern that we'll define in a library function
  \end{block}
  \begin{alertblock}{}
    \begin{itemize}
    \item \texttt{undefined} is a value of any type
    \item executing it yields a run-time error
    \end{itemize}
  \end{alertblock}
\end{frame}

%----------------------------------------------------------------------
\begin{frame}[fragile]
  \frametitle{The function map}
  \begin{block}{Definition}
\begin{verbatim}
-- map f [x1, x2, ..., xn] = [f x1, f x2, ..., fn]
map f []     = undefined
map f (x:xs) = undefined
\end{verbatim}
    (map is in the standard Prelude - no need to define it)
  \end{block}
  \begin{block}{Doubles in terms of map}
\begin{verbatim}
doubles xs = map double xs

double :: Integer -> Integer
double x = undefined
\end{verbatim}
  \end{block}
\end{frame}
%----------------------------------------------------------------------
\begin{frame}[fragile]
  \frametitle{The function filter}
  Produce a list by removing all elements 
  which do not have a certain property from 
  a given list: 

\begin{verbatim}
filter odd [1,2,3,4,5] == [1,3,5]
\end{verbatim}

  \begin{block}{Definition}
\begin{verbatim}
filter :: (a -> Bool) -> [a] -> [a]
filter p []       = undefined
filter p (x:xs) = undefined
\end{verbatim}
(filter is in the standard Prelude - no need to define it)
  \end{block}
\end{frame}
%----------------------------------------------------------------------

\begin{frame}
  \frametitle{Break Time --- Questions?}
  \begin{center}
    \tikz{\node[scale=15] at (0,0){?};}
  \end{center}
\end{frame}


\end{document}

%%% Local Variables: 
%%% mode: latex
%%% TeX-master: t
%%% End: 
